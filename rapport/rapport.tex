\documentclass[12pt]{article}
\usepackage[french]{varioref}
\usepackage[french]{babel}
\usepackage[T1]{fontenc}
\usepackage{lmodern}
\usepackage{graphicx} % Required for inserting images
\usepackage{float}
\usepackage{minted}
\title{Rapport projet de systéme}
\author{Leroy Florent et Salles Théo}
\date{Décembre 2023}
\begin{document}
\maketitle
\tableofcontents
\newpage
\section{Utilisation}
\section{Client}
\section{File synchronisée}
La file synchronisée est construite comme une bibliothéque avec un fichier
interface \textit{file\_sync.h} et un fichier implementation
\textit{file\_sync.c}.
\subsection{file\_sync.h}
Le module est composer de 4 fonction. Un fonction de création de la file
synchronisée, un qui libére les ressources allouées à la file et 2 autre
fonction pour ajouter
et retirer des éléments de cette file.
\subsection{file\_sync.c}
Toutes les informations de la file tel que les sémaphores, le buffer et les têtes de lecture 
et d'écriture sont conservées des un segment de mémoire partagé qui est conservées et
initialisé

\section{Lanceur}
\end{document}