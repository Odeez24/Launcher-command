\documentclass[12pt]{article}
\usepackage[french]{varioref}
\usepackage[french]{babel}
\usepackage[T1]{fontenc}
\usepackage{lmodern}
\usepackage{graphicx} % Required for inserting images
\usepackage{float}
\usepackage{minted}
\title{Rapport projet de systéme}
\author{Leroy Florent et Salles Théo}
\date{Décembre 2023}
\begin{document}
\maketitle
\tableofcontents
\newpage
\section{File synchronisée}
La file synchronisée est construite comme une bibliothèque avec un fichier
interface \textit{file\_sync.h} et un fichier implémentions.
\textit{file\_sync.c}.
\subsection{file\_sync.h}
Le module est composé de 4 fonctions. Une fonction de création de la file
synchronisée, une qui libère les ressources allouées à la file et 2 autres
fonctions pour ajouter
ou retirer des éléments de cette file.
\subsection{file\_sync.c}
Toutes les informations de la file tel que les sémaphores, le buffer de données et les
têtes de lecture
et d'écriture sont conservées dans un segment de mémoire partagé qui est créer
dans la fonction \textit{create\_file\_sync} avant l'initialisation des
sémaphores,
des têtes de lectures et du buffer.
Chaque action sur la file commence par l'ouverture du segment de mémoire
partagée
puis pour la fonction \textit{destroy\_file} on détruit toutes les sémaphores
puis on libère le segment de mémoire partagée. Pour les fonctions
\textit{defiler} et \textit{enfiler}
c'est un simple problème consommateur/producteur avec les sémaphores.
\section{Parseur}
Le module parsuer est une réutilisation du module \textit{analyse} donner lors du tp 2
avec des tests rajoutée notament sur les malloc.
\section{Lanceur de commande}
Le nom des tubes est définis par une normes choisis qui s'applique
entre le lanceur et le client. Cette norme veut que le nom d'un tube soit une
chaîne de caractère qui
définit l'utilité du tube suivit du pid du client afin que chaque client
possède des tube qui lui sont propres.
\begin{itemize}
  \item TUBE\_CLIENT\_: est le tube dans lequel le client écrit les
        informations
        pour le lanceur, il est crée par le client.
  \item TUBE\_RES\_CLIENT\_: est le tube dans lequel le resultat de la commande
        sera ecrit et envoyé au client, il est crée par le lanceur.
  \item TUBE\_ERR\_CLIENT\_: est le tube dans lequel une possible erreur lors
        de l'exécution de la commande sera écrit et envoyé au client, il est
        crée par le lanceur.
  \item BUF\_SIZE: Taille maximale en nombre de caractère des commandes avec
        leurs options.
  \item CMD\_SIZE: Taille maximale en nombre de  du nom de la commande
        (taille déterminer à partir du nombre de caractère de la plus grande
        commande
        linux avec une marge de sécurité).
  \item PID\_SIZE: Taille maximale en nombre de caractère du pid du client.
\end{itemize}
\section{Client}

\end{document}