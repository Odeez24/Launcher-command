\documentclass[12pt]{article}
\usepackage[french]{varioref}
\usepackage[french]{babel}
\usepackage[T1]{fontenc}
\usepackage{lmodern}
\usepackage{graphicx} % Required for inserting images
\usepackage{float}
\usepackage{minted}
\title{Rapport de projet Introduction aux Systèmes d'Exploitation}
\author{Leroy Florent et Salles Théo}
\date{Décembre 2023}
\begin{document}
\begin{titlepage}
    \vspace*{\stretch{1}}
    \begin{center}
        {\large\bfseries Rapport de projet \\[1ex]}
        {\huge\bfseries Introduction aux \\[0.25ex] Systèmes d'Exploitation
            \\[2ex]}
        {\LARGE\bfseries Lanceur de commandes \\[6.5ex]}
        {\large\bfseries LEROY Florent et SALLES Théo \\}
        \vspace{4ex}
        Rapport rédigé pour\\[5pt]
        \textit{Université de Rouen Normandie}\\[2cm]
        \textsc{\large Informatique}\\[12ex]
        \vfill
        Département d'informatique\\
        \vfill
        Décembre 2023
    \end{center}
    \vspace{\stretch{2}}
\end{titlepage}
\newpage
\tableofcontents
\newpage

\section{Introduction}
L'objectif de ce projet est de créer un lanceur de commandes dont les
commandes à exécuter sont passés à l'aide d'un client. De plus, le client
permet à l'utilisateur d'interagir directement avec les programmes exécutés. Le
tout prenant en compte le fait que plusieurs clients peuvent envoyer ces
commandes simultanément.

Ce projet se divise donc en 3 parties principales :
\begin{itemize}
    \item La file synchronisée, qui doit servir de stockage et de
          traitement des demandes d'exécutions commun aux clients et au lanceur de
          commandes, tout en gérant de manière synchrone l'enfilement et le défilement de
          données, à l'aide de sémaphores;
    \item Le client, qui doit traiter les demandes de l'utilisateur, créer
          les tubes servant d'entrées et de sorties standards à la commande et enfiler la
          demande d'exécution de commande dans la file;
    \item Le lanceur de commandes en lui-même, qui, à partir d'une demande
          de commande issu d'une file, modifie ses entrées et sorties standards pour
          correspondre aux tubes cités ci-dessus et exécuter la commande dans un nouveau
          thread.
\end{itemize}

\section{Manuel d'utilisation}

Pour utiliser le code réalisé dans ce projet correctement, les étapes
suivantes doivent être suivies rigoureusement et dans l'ordre.

Tout comportement non prévu issu du mauvais suivi de ces instructions ne
sont pas de notre responsabilité. De plus, les conséquences issus de ces
comportement sont entièrement de la faute de l'utilisateur.

\begin{enumerate}
    \item Récupérer toutes les dépendances de ce projet ( \texttt{gcc},
          \texttt{make} )
    \item Compiler les deux parties du code
          \begin{enumerate}
              \item Pour compiler le lanceur, exécuter la commande suivante :
                    \begin{center}

                        \texttt{make lanceur}
                    \end{center}
              \item Pour compiler le client, exécuter la commande suivante :
                    \begin{center}

                        \texttt{make client}
                    \end{center}
          \end{enumerate}
    \item Exécuter le lanceur avec la commande suivante : \begin{center}

              \texttt{./lanceur}
          \end{center}
    \item Envoyer une commande au lanceur à partir du lanceur avec la
          commande suivante ( remplacer les "\textbf{[CMDX]}" par les commandes à
          exécuter) : \begin{center}

              \texttt{./client [CMD1] | ... | [CMDN]}
          \end{center}
          Il est bon de remarqur que le client affichera la syntaxe attendue si
          exécuté avec aucun argument.
    \item Pour fermer le lanceur, deux possibilités :
          \begin{itemize}
              \item Récupérer le PID du lanceur avec la commande \texttt{ps} puis
                    le tuer avec la commande \texttt{kill [PID]}
              \item Tuer toutes les instances du lanceur avec la commande
                    \texttt{killall lanceur}
          \end{itemize}
\end{enumerate}

\section{File synchronisée}
Notre implémentation des files synchronisées est utilisable à partir d'une
bibliothèque \texttt{file\_sync}.

Son fonctionnement se base sur l'utilisation d'un mutex pour éviter
l'édition de la pile par un processus alors que celle-ci est activement
utilisée par d'autres processus. Nous pouvons aussi remarquer que
l'implémentation utilise l'algorithme solvant le problème du
producteur/consommateur, c'est-à-dire que tout défilement sur une pile vide ou
enfilement dans une file pleine provoquera un blocage, à l'aide d'une paire de
sémaphores.

De plus, pour permettre à chaque acteur (clients, lanceur, ...) d'accéder à
la file, et cela dans n'importe quelle situation (par exemple, en imaginant que
chaque acteur possède un répertoire actif différent) nous devons la placer dans
un segment de mémoire partagée, ce qui insinue que notre bibliothèque doit
permettre à l'utilisateur de gérer (créer, détruire) ces segments de mémoire
partagée.

Enfin, dans notre implémentation actuelle, nous avons codé la file de
manière à ce qu'elle puisse stocker des identifiants de processus (codée en C
sous forme de \texttt{pid\_t}, type synonyme au type \texttt{signed int} sous
Linux).

La bibliothèque \texttt{file\_sync} offre l'accès aux fonctions suivantes :

\subsection{\texttt{create\_file\_sync}}

\begin{center}
    \texttt{int create\_file\_sync(void)}
\end{center}

Cette fonction crée une file synchronisée et de la stocker dans une zone
mémoire partagée. Cette zone mémoire sera nommée selon la macroconstante
\texttt{NOM\_SHM}.
Renvoie \texttt{-1} en cas d'erreur, \texttt{0} sinon. Les erreurs peuvent
être causées par :
\begin{itemize}
    \item L'existence d'un segment de mémoire partagée ayant pour nom
          \texttt{NOM\_SHM} (il est à noter que ce cas de figure peut se produire si,
          lors d'une utilisation précédente de la bibliothèque, le segment n'a pas été
          supprimée correctement);
    \item Un manque d'espace suffisant pour créer le segment;
    \item Une erreur lors de la création des sémaphores.
\end{itemize}

\subsection{\texttt{destroy\_file}}
\begin{center}
    \texttt{int destroy\_file(void)}
\end{center}

Cette fonction supprime la file précédemment crée portant le nom
\texttt{NOM\_SHM} et libère les ressources associées.
Renvoie \texttt{-1} en cas d'erreur, \texttt{0} sinon. Les erreurs peuvent
être causées par :
\begin{itemize}
    \item L'absence d'un segment de mémoire partagée ayant pour nom
          \texttt{NOM\_SHM} (il est à noter que ce cas de figure peut se produire si,
          lors d'une utilisation précédente de la bibliothèque, le segment n'a pas été
          créé correctement);
    \item Une erreur lors de la destruction des sémaphores.
\end{itemize}

\subsection{\texttt{enfiler}}
\begin{center}
    \texttt{int enfiler(pid\_t donnee)}
\end{center}

Cette fonction enfile un identifiant de processus \texttt{donnee} dans la
file synchronisée. Si la file est pleine et/ou en cours d'utilisation, l'appel
sera bloquant jusqu'à ce que la file soit défilée au moins une fois et/ou que
la file soit libre/inutilisée.
Renvoie \texttt{-1} en cas d'erreur, \texttt{0} sinon. Les erreurs peuvent
être causées par :
\begin{itemize}
    \item L'absence d'un segment de mémoire partagée ayant pour nom
          \texttt{NOM\_SHM} (il est à noter que ce cas de figure peut se produire si,
          lors d'une utilisation précédente de la bibliothèque, le segment n'a pas été
          créé correctement);
    \item Une erreur lors de l'édition d'un des sémaphores.
\end{itemize}

\subsection{\texttt{defiler}}
\begin{center}
    \texttt{pid\_t defiler(void)}
\end{center}

Cette fonction défile depuis la file synchronisée un identifiant de
processus et la renvoie. Si la file est vide et/ou en cours d'utilisation,
l'appel sera bloquant jusqu'à ce que la file soit enfilée au moins une fois
et/ou que la file soit libre/inutilisée.
Renvoie \texttt{(pid\_t) -1} en cas d'erreur, l'identifiant défilé sinon.
Les erreurs peuvent être causées par :
\begin{itemize}
    \item L'absence d'un segment de mémoire partagée ayant pour nom
          \texttt{NOM\_SHM} (il est à noter que ce cas de figure peut se produire si,
          lors d'une utilisation précédente de la bibliothèque, le segment n'a pas été
          créé correctement);
    \item Une erreur lors de l'édition d'un des sémaphores.
\end{itemize}

\section{Parseur}
Le module parseur est une réutilisation du module \texttt{analyse} donnée
lors
du TP 2 qui contient ces deux fonctions.

\begin{itemize}
    \item \texttt{char **parseur\_arg(const char arg[])} : Prend une chaîne
          de caractère \texttt{arg} et la découpe en chaîne de caractère distincts selon
          les espaces.

    \item \texttt{void dispose\_arg(char *argv[])}: Libère les ressources
          allouées pour le tableau retourné par \texttt{parseur\_arg}.
\end{itemize}
Les fonctions n'ont uniquement subi comme modification des test rajoutés
sur les \texttt{malloc}.

\section{Lanceur de commande}
Le nom des tubes est défini par une norme choisie commune aux clients et au
lanceur. Cette norme veut que le nom d'un tube soit une chaîne de caractère qui
définit l'utilité du tube suivit du PID du client afin que chaque client
possède des tube qui lui soient propres. Tout les tubes sont créés par le
client.

\subsection{Macroconstantes}
\begin{itemize}
    \item \texttt{TUBE\_CL}: préfixe du tube dans lequel le client écrit
          les
          informations pour le lanceur.
    \item \texttt{TUBE\_RES}: préfixe du tube dans lequel le résultat de la
          commande sera écrit et envoyé au client.
    \item \texttt{TUBE\_ERR}: préfixe du tube dans lequel une possible
          erreur
          lors de l'exécution de la commande sera écrit et envoyé au client.
    \item \texttt{BUF\_SIZE}: Taille maximale en nombre de caractère des
          commandes
          avec leurs options.
    \item \texttt{CMD\_SIZE}: Taille maximale en nombre de du nom de la
          commande
          (taille déterminer à partir du nombre de caractère de la plus
          grande commande linux avec une marge de sécurité).
    \item \texttt{PID\_SIZE}: Taille maximale en nombre de caractère du pid
          du client.
\end{itemize}

\subsection{Structure \texttt{my\_thread\_args}}
La structure \texttt{my\_thread\_args} contient tous les arguments à
transmettre au thread.
Il est à noter qu'avec notre implémentation actuelle, seul le PID du cilent
est transmis. Cette structure est codée de telle manière à faciliter
l'amélioration du code, permettant d'ajouter aisément de nouveaux éléments
utiles aux threads.

\subsection{Bloc \texttt{main}}
Le principe du \texttt{main} est de créer un thread pour chaque demande
d'exécution de
commande.
Nous commençons par faire du lanceur un deamon, pour cela nous faisons un
processus fils (avec un \texttt{fork}) dans lequel on le dissocie de la session
avec la fonction \texttt{setsid}.
Puis on s'occupe de la gestion du signal pour terminer le lanceur. En
effet, comme le lanceur est un daemon, l'utilisateur va devoir lui envoyer un
signal pour l'arrêter. On admet ici que l'utilisateur utilisera les commandes
\texttt{kill  [PID]} ou \texttt{killall [NOM]} pour arrêter le lanceur. Comme
ces commandes envoient le signal \texttt{SIGTERM} par défaut, nous devons
associer la réception de ce signal avec :
\begin{itemize}
    \item La destruction de la file synchronisée avec la fonction
          \texttt{destroy\_file}
    \item La fermeture du programme en lui-même.
\end{itemize}

Ensuite, tant que nous pouvons défiler un PID depuis la file synchronisée,
on crée un thread dans lequel on passe comme argument le PID du client.

\subsection{Fonction \texttt{run}}
La fonction de lancement du thread crée un fils dans lequel on ouvre les 3
tubes lui permettant de communiquer avec le client. Puis il lit les
informations donné par le client sur le tube prévu à cette effet. Ensuite on
redirige les deux sorties standards (\texttt{stdout} et \texttt{stderr}) vers
les deux tubes ouverts en écriture. On peut ensuite appeler la fonction
\texttt{parseur\_arg} du module \texttt{parseur} afin de récupérer la commande
ainsi que chaque option de celle-ci. On finit par utiliser la fonction
\texttt{execvp} afin d'exécuter la commande demandée. Le père lui ne fait que
attendre son fils afin
de libérer les ressources proprement.

\section{Client}
L'objectif du client est de faire parvenir au lanceur toutes les
informations transmises par l'utilisateur à travers les arguments passés lors
de son appel. Ces informations sont passées par deux canaux différents :
\begin{itemize}
    \item L'envoi du PID du client au lanceur se fait à travers la file
          synchronisée;
    \item L'envoi de la commande à exécuter par le lanceur se fait à
          travers le tube portant le nom \texttt{TUBE\_CL}
\end{itemize}
On commence par générer le nom des tubes avant de créer ces tubes. Ensuite
on
envoie le PID du client dans la file afin que le lanceur puisse ouvrir les
tubes.
Maintenant on ouvre le tube de communication entre le client et le lanceur
et on y écrit la commande à exécuter.
Nous pouvons ensuite ouvrir les tubes de communication entre le lanceur et
le client \texttt{TUBE\_CL}. Enfin, le client essaye de lire sur chaque tube la
sortie de la commande (à travers les tubes \texttt{TUBE\_RES} et
\texttt{TUBE\_ERR}) et l'affiche sur la sortie correspondante.

\section{Difficultés rencontrées}
\begin{itemize}
    \item Nous n'avons pas réussi à implémenter l'utilisation des tubes
          lors du passage
          des commandes.
    \item Nous avons rencontré des difficultés pour formater/parser la
          commande dans le lanceur
          afin de pouvoir utiliser \texttt{execvp}.
    \item Dans le client, nous avons eu des difficultés liés à la
          récupération des
          informations issus des tubes de réponses. En effet, nous pouvions soit
          rediriger les sorties du client ou plus simplement lire dans les tubes puis
          écrire dans les sorties. Nous avons au final opté pour la seconde solution.
    \item La gestion du "timing" entre le client et le lanceur fut
          compliqué. En effet, nous avions dû constamment changer l'emplacement des blocs
          de code permettant les créations/lectures/écritures des différents tubes afin
          d'éviter une lecture/écriture sans écrivain/lecteur, bloquant indéfiniment
          l'exécution de notre code.
    \item L'aspect ouvert de ce projet a mené à de nombreux désaccords,
          chaque personne ayant participé au projet ayant une interprétation différente
          de l'énoncé.
\end{itemize}

\section{Remerciements}
Nous tenons à remercier tout particulièrement \begin{itemize}
    \item \textbf{HADDAG
              Édouard}
    \item \textbf{RENAUX
              VERDIÈRE Théo}
    \item \textbf{DUMONTIER
              Louis}
\end{itemize} pour leur aide dans la compréhension et la réalisation de ce
projet.

\end{document}